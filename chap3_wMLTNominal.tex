\section{Selectively updating value functions of nominal attributes}
As seen in Section \ref{sec:valueFunc}, the value of $\gamma$ used to update the value functions of nominal attributes is constant in each cycle.
Instead, we propose an alternative approach in which the value of $\gamma$ varies according to the number of alternatives the user rejected while choosing a particular attribute value. 
Higher the number of rejected alternatives, higher is the value of $\gamma$.
The intuition for varying the value of $\gamma$ is as follows: Consider the case when there are five PCs displayed to the user and the manufacturers of the five PCs are "Compaq", "HP", "Apple", "Dell" and "Toshiba".
If the user selects the PC with "Compaq" as the manufacturer, we can see that he has accepted "Compaq" and rejected four other manufacturers.
Thus, we infer that the user has a strong preference for "Compaq" PCs.
On the other hand, if all the five PCs have a screen size of 15 inches and the user selects the PC with screen-size of 15 inches, we cannot really infer whether the user has a strong preference for 15 inch PCs.
Generalizing the examples above, if $k$ is the value of the attribute N selected in a cycle and $R$ is the list of values of attribute $N$ of the top-K products other than the selected product, we define
%
\begin{equation}
\gamma = \frac{\#\: of\: alternatives\: to\: k\: in\: R}{|R|}
\end{equation}
%
If attribute $N$ = "Manufacturer"; $\gamma$ = 1 when manufacturer of all the remaining products is different from the selected product's manufacturer($M$) and also different from each other; meaning that the user has a strong preference for $M$.
$\gamma$ = 0 when manufacturer of all remaining products is same as the selected product's manufacturer($M$).
This is the case when we cannot infer whether the user has a strong preference for $M$.
This is similar to weighted MLT described in \cite{comparisonbr}.
The value of $\gamma$ for each attribute when the product \textit{P3} is selected, is illustrated in Table \ref{tab:wMLT}.
Varying the value of $\gamma$ according to the other products' attribute values results in a significant improvement in the number of interaction cycles.



\begin{table}
\renewcommand{\arraystretch}{1.5}
 \centering
 \begin{tabular}{|l| l l l l l |l|}
  \hline \hline
   Feature & P1 & P2 & \textbf{P3} & P4 & P5 & $\gamma$ \\
  \hline
  Manufacturer & Dell & Apple & \textbf{Compaq} & HP & Toshiba & 1.0 \\
  Type & Laptop & Laptop & \textbf{Laptop} & Laptop & Laptop & 0 \\
  Processor & Core-i3 & AMD-E1 & \textbf{Core-i7} & AMD-E1 & Core-i3 & 0.5\\
  Screen-size & 15 & 13.3 & \textbf{15} & 15 & 15 & 0.25\\
  \hline \hline
 \end{tabular}
 \caption{Values of $\gamma$ when P3 is selected by the user}
 \label{tab:wMLT}
\end{table}

