\section{Varying the level of diversity}
In Section \ref{sec:div}, we have discussed a method of introducing diversity in every cycle.
At the beginning of a recommendation session, an average user does not have a good understanding of the product space and trade-offs that exist between different product attributes.
But after interacting with the recommender system for a few cycles, he develops a good understanding of the product spac.
Therefore, it is a good practice to introduce diversity in the beginning of a recommendation session so that user will develop a better understanding of the product space and then show targeted recommendations after a few cycles after his preferences have stabilized.

We propose a new approach where diversity is varied adaptively according to whether user's preferences have stabilized.
Consider an attribute 'Price' (All products are in the camera domain).
If the maximum difference between prices of previous $K$(=3) products selected by the user is less than a pre-defined threshold, we assume that the user is satisfied with this particular product price and we promote cases that have a price closer to the average price of the $K$ products in the next cycle.
%Consdiering camera dataset, if the difference between 'resolution' of each of the previous $K$(=3) selected products is less than a certain threshold,
If the maximum difference between the resolutions of the previous $K$(=3) products is greater than the pre-defined threshold, we assume that user's preference for the 'resolution' attribute has not stabilized yet. 
Hence, we try to display products that are diverse to each other in terms of 'resolution' attribute.
The numeric attributes for which we assume that user's preference has stabilized are classified into the set $simA$.
The other numeric attributes are classified into the set $divA$.
The function $Quality(p, R, PM)$ is modified as follows:

\begin{algorithm}[ht]
  \SetKwInOut{Input}{input}\SetKwInOut{Output}{output}
  \DontPrintSemicolon
  %\Input{$i$, $R$, $PM$}

  $ret \gets \alpha \times utility(i, PM)$; \\
  $simA, divA$ = classifyAttributes(); \\
  $avg$ = previousKProductAttributeAverages(); \\
  $tmp \gets 0$;\\
  \For{ each attribute $x_i$ in $simA$ }{
      $tmp$ += (sim($avg(x_i)$, $p(x_i)$));\\
  }
  \For{ each attribute $x_i$ in $divA$ }{
      $tmp += \frac{\sum_{r_j \in R} (1-sim(p(x_i),r_j(x_i)))}{|R|}$;\\
  }

  $tmp = (1-\alpha) \times \frac{tmp} {\lvert simA\rvert + \lvert divA\rvert}$\\
  \Return $ret + tmp$ ;\\
  \caption{Quality(p, R, PM)}
  \label{algo:quality2}
\end{algorithm}

