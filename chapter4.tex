\chapter{Experimental Results}
\label{chap:results}
\section{Experimental Methodology}


%Add the introduction sentence, we compare so-and-so with standard MAUT
We use two standard datasets, \textbf{Camera} and \textbf{PC}, in our experiments.
Both the datasets are available for download at \cite{datasets}.
Cases with missing values have been removed from the datasets.
After removing the cases with missing values, the Camera dataset contains 173 cameras with 10 attributes and the PC dataset contains 120 PCs with 8 attributes.
A typical PC in the PC dataset is shown in Table \ref{tab:pc} and a typical camera in the Camera dataset is shown in Table \ref{tab:camera}.


\begin{table}
\caption{A typical PC in the PC dataset}
\centering
\renewcommand{\arraystretch}{1.2}
\label{tab:pc}

\begin{tabular}{|l|l|}
\hline
Manufacturer & Apple \\
\hline
Processor Type & PowerPC G3 \\
\hline
Processor Speed(MHz) & 600 \\
\hline
Monitor (Inches) & 15 \\
\hline
Type & Laptop \\ 
\hline
RAM (MB) & 512 \\
\hline
Drive Capacity(GB) & 40 \\
\hline
Price (\$) & 986\\
\hline
\end{tabular}
\end{table}

\begin{table}
\caption{A typical camera in the Camera dataset}
\centering
\renewcommand{\arraystretch}{1.2}
\label{tab:camera}

\begin{tabular}{|l|l|}
\hline
Manufacturer & Sony \\
\hline
Model & DSC-T11 \\
\hline
Price(\$) & 383\\
\hline
Format & Ultra-compact\\
\hline
Resolution (MP)  &5\\
\hline
Optical Zoom (X) &3 \\
\hline
Digital Zoom (X) &4\\
\hline
Weight (grams) &230\\
\hline
Storage Type & Memory Stick\\
\hline
Storage Included (MB)& 32\\
\hline
\end{tabular}
\end{table}

To evaluate the algorithms in offline setting, we simulate an artificial user who interacts with the recommender system.
A product is first selected from the case base and a random subset of it's features is used to formulate the simulated user's query at the beginning of the recommendation session.
This product is now the target for the recommender system and the recommendation session terminates when this product appears in the list of $K$ products shown to the user. 
Subsets of 1, 3 and 5 features are referred to as Q1, Q3 and Q5 respectively.
Queries in Q1 category correspond to users who have very limited domain knowledge.
Queries in Q5 category correspond to users who have a very clear idea about what products they exactly want.
For both PC and Camera datasets, we generate queries of the type Q1, Q3 and Q5.
The same set of queries has been used for all the experiments done below.
Each product in the case-base is set as the target product 10 times and this is repeated for all products in the case-base.
Hence a total of 1730 and 1200 queries for each of the three categories Q1, Q3 and Q5 have been generated for Camera and PC datasets respectively.
The evaluation of all the algorithms described in Sections \ref{sec:div} to \ref{sec:additive} are performed in two scenarios described in Section \ref{sec:focus} and \ref{sec:noisy}, where the artifical user interacts with the system in two different ways.
The complete implementation for all the algorithms described in Sections \ref{sec:div} to \ref{sec:additive} is available at 
%\url{https://github.com/abharath27/MAUTNew}.
\cite{implementation}.

\section{Highly Focused Recommendation Framework}
\label{sec:focus}
Evaluation in Highly Focused Recommendation Framework is the same way of evaluation described in Section \ref{sec:offline}.
In this scenario we assume that the user is relatively sure of his preferences and hence chooses the critique string that is maximally compatible with the target product in each cycle.
The notion of selecting the most compatible critique string is shown in Figure \ref{fig:focus}

%This will help  us to illustrate the noisy framework better.
\begin{figure}[h]
  \centering
  \captionsetup{justification=centering}
    \includegraphics[width=0.5\textwidth]{figures-bharath/focus.pdf}
  \caption{Numbers in the boxes represent the compatibilities of each of the five critique strings with the target product. Simulated user selects the most compatible critique string (red) in each cycle.}
\label{fig:focus}
\end{figure}

\begin{figure}[h]
  \centering
  \captionsetup{justification=centering}
    \includegraphics[width=0.5\textwidth]{figures-bharath/noisy.pdf}
  \caption{Simulated user selects sub-optimal critique strings due to the introduction of noise}
\label{fig:noisy}
\end{figure}

\section{Noisy Framework}
\label{sec:noisy}
In this scenario, the simulated user does not select the most optimal critique string during each cycle.
This kind of evaluation is similar to the evaluation procedure described in \cite{suggestion}.
Noise is introduced into the process by varying the compatibility scores of the critique strings within some threshold. 
In our experiments, we have used a noise level of 10\%, i.e, the compatiblity scores can be changed by upto +/-10\% of their actual values.
Due to the introduction of noise, the uesr makes sub-optimal choices in each cycle as shown in Figure \ref{fig:noisy}.
In the experiments, the simulated user selects a sub-optimal critique string for 28\% times on an average in noisy setting.
The average number of interaction cycles in this scenario is slightly higher in this scenario compared to the number of cycles in highly focused recommendation framework because of sub-optimal choices made by the simulated user.


\section{Results}
\subsection{Diversity enhancing algorithms}
\label{sec:div_results}
The algorithm described in Section \ref{sec:div} (abbreivated as DIV), introduces diversity among critique strings in every cycle. 
The algorithm described in Section \ref{sec:div2} (abbreivated as DIV2), introuduces diversity according to the extent to which the user's preferences have stabilized.
The results for the algorithms DIV and DIV2 in both optimal and noisy settings for PC and Camera datasets are summarized in the Figures \ref{fig:div_camera_opt} to \ref{fig:div_pc_noisy}.
For the queries of category Q1 on the in Figure \ref{fig:div_camera_opt}, the implementation of standard MAUT based recommendation takes 9.21 cycles on an average to reach a target product.
DIV and DIV2 take 7.82 and 6.82 cycles on an average to reach the target product.
DIV and DIV2 achieve a reduction of 22.6\% and 27.8\% respectively in the average number of cycles.
The average number of cycles in MAUT is 10.35 in noisy setting, which is 12\% higher than the number of cycles in noisy setting.



\begin{figure}[h]
\centering
\begin{minipage}{.45\textwidth}
  \centering
  \includegraphics[width=1\linewidth]{figures-bharath/div_camera_opt}
  \caption[]{Average number of interaction cycles on Camera dataset - optimal user model}
  \label{fig:div_camera_opt}
\end{minipage}%
\;\;\;\;\;\;
\begin{minipage}{.45\textwidth}
  \centering
  \includegraphics[width=1\linewidth]{figures-bharath/div_pc_opt}
  \caption[]{Average number of interaction cycles on PC dataset - optimal user model}
  \label{fig:div_pc_opt}
\end{minipage}
\end{figure}

\begin{figure}[h]
\centering
\begin{minipage}{.45\textwidth}
  \centering
  \includegraphics[width=1\linewidth]{figures-bharath/div_camera_noisy}
  \caption[]{Average number of interaction cycles on Camera dataset - noisy framework}
  \label{fig:div_camera_noisy}
\end{minipage}%
\;\;\;\;\;\;
\begin{minipage}{.45\textwidth}
  \centering
  \includegraphics[width=1\linewidth]{figures-bharath/div_pc_noisy}
  \caption[]{Average number of interaction cycles on PC dataset - noisy framework}
  \label{fig:div_pc_noisy}
\end{minipage}
\end{figure}

\subsection{Algorithms that combine similarity with utility}
In Algorithm \ref{algo:addPref} (ADDPREF), we have promoted products that have a critique pattern similar to the product selected by the user in previous cycle.
In Algorithm \ref{algo:addPref2} (ADDPREF2), an extension to ADDPREF, we consider critique overlap of a product with all the products that have been selected so far.
The performance for both the algorithms are shown in Figures \ref{fig:addPref_camera_opt} to \ref{fig:addPref_pc_noisy}.
In highly focused recommendation framework, ADDPREF and ADDPREF2 give an average improvement of 31.1\% and 33.8\% respectively.

\begin{figure}[h]
\centering
\begin{minipage}{.45\textwidth}
  \centering
  \includegraphics[width=1\linewidth]{figures-bharath/addPref_camera_opt}
  \caption[]{Average number of interaction cycles on Camera dataset - optimal user model}
  \label{fig:addPref_camera_opt}
\end{minipage}%
\;\;\;\;\;\;
\begin{minipage}{.45\textwidth}
  \centering
  \includegraphics[width=1\linewidth]{figures-bharath/addPref_pc_opt}
  \caption[]{Average number of interaction cycles on PC dataset - optimal user model}
  \label{fig:addPref_pc_opt}
\end{minipage}
\end{figure}

\begin{figure}[h]
\centering
\begin{minipage}{.45\textwidth}
  \centering
  \includegraphics[width=1\linewidth]{figures-bharath/addPref_camera_noisy}
  \caption[]{Average number of interaction cycles on Camera dataset - noisy framework}
  \label{fig:addPref_camera_noisy}
\end{minipage}%
\;\;\;\;\;\;
\begin{minipage}{.45\textwidth}
  \centering
  \includegraphics[width=1\linewidth]{figures-bharath/addPref_pc_noisy}
  \caption[]{Average number of interaction cycles on PC dataset - noisy framework}
  \label{fig:addPref_pc_noisy}
\end{minipage}
\end{figure}

In Section \ref{sec:sim}, we have proposed a modification(SIM) where we introduce the products that are most similar to user's initial query.
This simple modification actually produces a very high improvement in the number of interaction cycles.
SIM reduces the average number of cycles by 32.6\% on an average.
SIM when combined with ADDPREF2, reduces the average number of cycles even further. 
The reduction is 40.3\% on average in this case.
The results obtained in the two scenarios for Camera and PC datasets have been shown in Figures \ref{fig:sim_camera_opt} to \ref{fig:sim_pc_noisy}.

\begin{figure}[h]
\centering
\begin{minipage}{.45\textwidth}
  \centering
  \includegraphics[width=1\linewidth]{figures-bharath/sim_camera_opt}
  \caption[]{Average number of interaction cycles on Camera dataset - optimal user model}
  \label{fig:sim_camera_opt}
\end{minipage}%
\;\;\;\;\;\;
\begin{minipage}{.45\textwidth}
  \centering
  \includegraphics[width=1\linewidth]{figures-bharath/sim_pc_opt}
  \caption[]{Average number of interaction cycles on PC dataset - optimal user model}
  \label{fig:sim_pc_opt}
\end{minipage}
\end{figure}

\begin{figure}[h]
\centering
\begin{minipage}{.45\textwidth}
  \centering
  \includegraphics[width=1\linewidth]{figures-bharath/sim_camera_noisy}
  \caption[]{Average number of interaction cycles on Camera dataset - noisy framework}
  \label{fig:sim_camera_noisy}
\end{minipage}%
\;\;\;\;\;\;
\begin{minipage}{.45\textwidth}
  \centering
  \includegraphics[width=1\linewidth]{figures-bharath/sim_pc_noisy}
  \caption[]{Average number of interaction cycles on PC dataset - noisy framework}
  \label{fig:sim_pc_noisy}
\end{minipage}
\end{figure}

\subsection{Results for SELNOMINAL and SELMAUT}

\begin{figure}[h]
\centering
\begin{minipage}{.45\textwidth}
  \centering
  \includegraphics[width=1\linewidth]{figures-bharath/sel_camera_opt}
  \caption[]{Average number of interaction cycles on Camera dataset - optimal user model}
  \label{fig:sel_camera_opt}
\end{minipage}%
\;\;\;\;\;\;
\begin{minipage}{.45\textwidth}
  \centering
  \includegraphics[width=1\linewidth]{figures-bharath/sel_pc_opt}
  \caption[]{Average number of interaction cycles on PC dataset - optimal user model}
  \label{fig:sel_pc_opt}
\end{minipage}
\end{figure}

\begin{figure}[h]
\centering
\begin{minipage}{.45\textwidth}
  \centering
  \includegraphics[width=1\linewidth]{figures-bharath/sel_camera_noisy}
  \caption[]{Average number of interaction cycles on Camera dataset - noisy framework}
  \label{fig:sel_camera_noisy}
\end{minipage}%
\;\;\;\;\;\;
\begin{minipage}{.45\textwidth}
  \centering
  \includegraphics[width=1\linewidth]{figures-bharath/sel_pc_noisy}
  \caption[]{Average number of interaction cycles on PC dataset - noisy framework}
  \label{fig:sel_pc_noisy}
\end{minipage}
\end{figure}

\input{chap4_history}
\input{chap4_marketEq}
\input{chap4_additive}
