\section{Considering history of user selected products}
\label{sec:hist}
%In Figure \ref{fig:maut}, $ref$ is the product selected by the user in previous interaction cycle.
As seen in Section \ref{sec:maut}, after the user selects a critique string, preference model is updated only based on the product corresponding to this critiquing string.
Instead, we now maintain a list of products corresponding to the critique strings selected by the user in previous interaction cycles and update the model according to the weighted average of these products' attributes.
For example, if the prices of the products selected by the user in the first four cycles are \$250, \$200, \$200 and \$400.
If he chooses a product with a price \$800 in the fifth cycle, the standard MAUT algorithm will make \$800 as the preferred value of 'price' attribute in it's user model and divide by weight of the price attribute by $\beta$, because the price of the reference product is \$350.
However, the user might have most likely selected the product with price \$800 because he likes other features of the product.
When this product becomes the reference product in the next cycle, the user will most likely choose "Lower Price" critique because all his previous preferrred products have a price around \$400.
Instead of making \$800 as the preferred price, we compute the weighted average ($wa$) of all previous product prices and make that as the preferred price and update the weight of 'price' attribute based on the value of $wa$.
%In general, the weights of numeric attributes will be updated based on weighted average of previously selected products' attribute values.
In our implementation, we considered the weight associated with $i^{th}$ product is $\frac{1}{n}$. 
Hence the preferred price for the given example will be:
\begin{equation}
wa = \frac{250+200+200+400+800} {5} = 370
\end{equation}
The value of new preferred price(\$370) is less than the old preferred price (\$400).
Hence, the weight of the price attribute is multiplied by a factor of $\beta$ and lower priced products are promoted in the next cycle instead of higher priced ones.
Note that this strategy is applicable only to numeric attributes.
The value functions of nominal attributes are updated in the standard way as described in Section \ref{sec:valueFunc}.
Algorithm \ref{algo:history} will referred to as \textbf{HIST} in the subsequent sections.
The function UpdateModel($PM$, $ref$) is updated as follows:


%Before that put an equation with a sigma in that...
%Give an example here...
%That would be like putting the last nail of the coffin to settle things down

\begin{algorithm}[ht]
  \SetKwInOut{Input}{input}\SetKwInOut{Output}{output}
  \DontPrintSemicolon
  %\Input{$PM$, $IS$}

  $R \gets \{\}$\\
  $CB' \gets IS$\\
  \For{each attribute $x_i$ }{
      $ref[x_i] = 0$;\\
      $ws = 0$;\\
      \For{each product p $\in$ refL }{
          $ref[x_i] +=  hw_i \times p[x_i] $;\\
          $ws += hw_i$;\\
      }
      $ref[x_i] = ref[x_i]/ws$; \\
  }
  \For{each attribute $x_i$ in ref }{
      $[pv_i, pw_i] \leftarrow PM on x_i$;\\
      \If {$V(x_i) \geq pv_i$} {
          $pw_i' = pw_i \times \beta $;\\
      }
      \Else {
          $pw_i' = pw_i / \beta $;\\
      }
      $PM' \leftarrow [V(x_i), pw_i']$;\\
  }
  \Return $PM$;\\

  \caption{UpdateModel(PM, refL)}
  \label{algo:history}
\end{algorithm}
