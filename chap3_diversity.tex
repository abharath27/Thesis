\section{Diversity in Critiques}
\label{sec:div}
In a live user-study conducted by \cite{aprioriUserStudy} to compare Apriori based critiquing with unit critiques, a number of trialists complained that their options were frequently limited because the compound critiques were too similar to each other.
As shown in Figure \ref{fig:beforeDiv} the same problem exists in the MAUT based critiquing also.
In the first few cycles, an average user does not have a good understanding of the product space and also different trade-offs that exist between product features.
Diverse critiques help educate a user about the product space relatively quickly and hence can lead to efficient recommendation sessions.
But in an attempt to make the critiques diverse, we will have to choose sub-optimal products as the top $K$ products in a cycle.
So there is also a risk of having prolonged sessions because of sub-optimal products being chosen into the list of top $K$ products in every cycle.
The default strategy for selecting the short-list of $K$ products is to select $K$ products that have highest utility.
Instead, we use a variant of \textit{\textbf{Bounded Greedy Selection}} algorithm described in \cite{boundedGreedy} to select the top $K$ products.
The main idea of this approach is that we select products with maximum utility while minimizing the average similarity to the cases/compound critiques selected so far.
A metric that is similar to Equation \ref{eq:quality} is used to compute $Quality$ scores of products in each cycle.
\begin{equation}
\label{eq:quality}
Quality(c, P) = \alpha * utility(c) + (1-\alpha)*critiqueSim(c, P)
\end{equation}

The function \textit{GenCritiqueItems(PM, IS)} is modified as follows:
\begin{algorithm}[ht]
  \SetKwInOut{Input}{input}\SetKwInOut{Output}{output}
  \DontPrintSemicolon
  %\Input{$PM$, $IS$}

  $R \gets \{\}$\\
  $CB' \gets IS$\\
  \For{ $i\gets0$ \KwTo $k$ }{
    Sort $CB'$ by $Quality(p, R, PM)$ for each case $p$ in IS; \\
    $R \gets R + First(CB')$;\\
    $CB' \gets CB' - First(CB')$;\\
  }
  \Return R
  \caption{GenCritiqueItems(PM, IS)}
  \label{algo:div}
\end{algorithm}

\begin{algorithm}[ht]
  \SetKwInOut{Input}{input}\SetKwInOut{Output}{output}
  \DontPrintSemicolon
  %\Input{$i$, $R$, $PM$}

  $retVal \gets \alpha \times utility(p, PM)$; \\
  \If {$R == \{\}$} {return retVal;}
  \Else {
      $disSim \gets \frac{\sum_{r_j \in R} (1-critiqueSim(p,r_j))}{|R|}$;\\
      $retVal += (1-\alpha) \times disSim$;\\
  }
  \Return retVal
  \caption{Quality(p, R, PM)}
  \label{algo:quality}
\end{algorithm}

$critiqueSim(a, b)$ returns the extent of overlap between the individual attribute directions of products $a$ and $b$.
We get the best results when $\alpha = 0.5$.
Introducing diverse critiques in every cycle results in significant improvement in the number of interaction cycles and it also improves user experience.



\begin{figure}
\centering
\begin{minipage}{.45\textwidth}
  \centering
  \includegraphics[width=1\linewidth]{figures-bharath/diversity1.jpg}
  \caption{Before diversifying critique strings}
  \label{fig:beforeDiv}
\end{minipage}%
\;\;\;\;\;\;
\begin{minipage}{.45\textwidth}
  \centering
  \includegraphics[width=1\linewidth]{figures-bharath/diversity2.jpg}
  \caption{After diversifying critique strings}
  \label{fig:afterDiv}
\end{minipage}
\end{figure}

