\abstract
\vspace*{24pt}

Most commercial recommender systems in practice use collaborative filtering (CF) techniques that rely heavily on user-ratings to make recommendations. However, CF may not perform well in high-risk product domains like cars, cameras, houses etc. where there a low number of ratings.Knowledge based Recommenders are used to provide recommendations in these scenarios.
In these domains, users often want to define their requirements explicitly - "The maximum price of PC should be \textit{x} and HDD capacity should be atleast \textit{500 GB}." and engage in an interaction with the system.
Thus, the recommendation process of a knowledge based recommender is highly interactive, and thus they are also characterized as \textit{conversational recommender systems}. 
Conversational recommender systems mimic the kind of dialog that takes place between a customer and shopkeeper involving multiple interactions and where the user can give feedback at every interaction.
\textit{Critiquing} is a popular form of feedback in conversational recommendation systems.

Dynamic generation of appropriate compound critiques in each cycle is a critical issue for critique-based conversational recommender systems.
In earlier research, Apriori algorithm and MAUT (Multi Attribute Utility Theory) based generation of compound critiques have been proposed.
MAUT based recommendation has been shown to be slightly superior to Apriori Algorithm based recommendation in offline experiments and live user studies.
"Average number of interaction cycles per recommendation session" is a measure that is often used to measure the efficiency of a recommendation algorithm. 
Lower the number of cycles, better is the performance of the algorithm. 
In this project, we propose several modifications to the MAUT based generation of compound critiques and report the improvements in performance caused by each of these modifications.






%Furthermore, a user's taste for products in these domains keeps changing with time. For example, if a user gave a rating of 5 stars to a Pentium-II computer 6 years ago, using that rating in making new recommendations will be in-appropriate.
